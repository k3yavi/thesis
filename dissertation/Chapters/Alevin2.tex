% Chapter 1

\chapter{Alevin 2} % Main chapter title

\label{alevin2} % For referencing the chapter elsewhere, use \ref{Chapter1} 

%----------------------------------------------------------------------------------------

\section{Background}

RNA-seq quantification has been shown as an important tool to explore the genome-wide
quantification of the gene expressions for both bulk and single cell RNA-seq experiments.
DscRNA-seq experiments have been especially useful, as with the advancement in droplet-based 
sequencing technologies (~\citep{dropseq, indrop, tenx}), it is able to generate data with high 
quantitative accuracy, sensitivity, and throughput. In ~\Cref{alevin}, we discussed a 
principled framework for generating cell wise gene-expression estimate given the read sequences
from a dcsRNA-seq experiment. We showed how discarding gene-multimapping reads by all the other dscRNA-seq quantification pipelines leads to biased expression estimates. 
Specifically, after the UMI resolution phase, the ambiguous reads with more than 1 gene label are assigned to
genes based on an expectation-maximization method. Although the framework works well in most of the cases,
however, in situations where there is no unique read evidence to confidently disambiguate the read assignment
among the genes, we call tier-3 estimates, the likelihood estimator uniformly divides the reads across the genes in the equivalence class label. In this study we propose the idea of information sharing across closely 
related cells using bayesian prior for specifically improving the tier-3 estimates. Intuitively speaking, 
since among the cells with similar cell-types we expect relatively similar gene expression, we propose that 
sharing information (in the form of expression) across neighboring cells improves the quantification 
estimates of individual cell. We show that the our idea of information sharing using bayesian 
prior to improve the quantification can be successfully applied to a dscRNA-seq experiment and improves the 
quantification estimates.

\section{Material and Methods}
\label{sec:alv2_methods}

\subsection{Variational Bayesian dscRNA-seq quantification}
Similar to salmon's ~\citep{salmon} collapsed variational bayesian optimization 
objective our aim is to quantify the expression, given a set of known genes $\mathcal{G}$ and a set
of gene-level equivalence classes with their UMI counts, generated post UMI deduplication using alevin's 
framework, as $\mathcal{E}$. The previous generation of alevin takes a maximum-likelihood based approach to 
optimize the gene-level ambiguous read assignment objective, however, it lacks the ability to utilize the 
confidence from neighboring cells. Since high level of sparsity is an inherant property of a dscRNA-seq
experiment and due to the random process of capturing the RNA molecule, in expectation, molecule sampling noise wouldn't be uniformly bad across all cells. We expect that sharing the confidence in the expression estimate across cells can be particularly effective in improving the cell-level expression. 

Salmon is a bulk RNA-seq quantification tool which uses bayesian priors to improve the 
quantification estimates by its online learning phase. We use the relaxed version of the algorithm 
to improve the estimates of alevin. Specifically, if we define gene-UMI count 
assignment matrix as $\mathcal{Z}$, where based on $\mathcal{E}$, $z_{ij}=1$ if UMI $j$ is derived from 
gene $i$. We also define the probability of generating a molecule from a particular gene as $\rho$ (analogous to 
nucleotide fraction in salmon model); we can write the probability of observing a set of deduplicated 
UMIs $\mathcal{U}$ as follows:

\begin{equation}
	\Pr\{\mathcal{U} | \mathcal{Z},\mathcal{G}\} = 
	\prod_{j=1}^{N}\sum_{i=1}^{M}\Pr\{ g_i | \rho \} \cdot \Pr\{ u_j | g_j, z_{i,j} = 1 \}
\end{equation}
 
 where $|\mathcal{U}| = N$ is the number of total molecules in the experiment (i.e. the number of 
 deduplicated UMIs ) and $|\mathcal{G}| = M$ is the number of total genes.
 
 In this study we take the Bayesian approach for the gene-expression estimation i.e. instead of 
 seeking the maximum-likelihood estimates we infer the posterior distribution of $\rho$. 
 This posterior distrbution can be defined as:

\begin{equation}
	\Pr \{ \rho | \mathcal{U} \mathcal{G} \} 
	\propto \sum_{\mathcal{Z}} \Pr\{ \mathcal{U} | \mathcal{G} \mathcal{Z} \}
	\cdot \Pr\{ \mathcal{Z} | \rho \} \cdot \Pr\{ \rho \}
\end{equation}

where both $\Pr\{ \mathcal{U} | \mathcal{G} \mathcal{Z} \}$ and $\Pr\{ \mathcal{Z} | \rho \}$ are 
a data dependant observables and can be estimated as defined in ~\citet{salmon}. The novelty of our method
comes with setting the prior term $\Pr\{ \rho \}$, specifically for dscRNA-seq data. We perform
within sample anchoring using Seurat3 (details in ~\Cref{subsec:anchor}) and use the gene expression
estimates from the nearest cells (with anchoring score >0.5) to generate cell-specific prior.

\subsection{Cellular Barcode Anchoring}
\label{subsec:anchor}
In ~\Cref{alevin} we defined tier 3 estimates as specifically the estimates which have reads assigned 
to gene ambiguous labels and they don't have any uniquely assignable read evidence in their equivalence 
class network. To disambiguate the gene assignment we use the 
information from the cells with similar type within the sample. To find similar cells for sharing 
the confidence in their gene expression estimates we use Seurat ~\citep{seurat3} based cellular
barcode anchoring scheme. In Seurat, ~\citet{seurat3} proposed an anchoring scheme in which they 
defined a framework to connect two experiments based on the similarity in their cell's gene expression 
pattern. They first find nearest neighbor using l2 distance of both inter and intra datatsets to generate 
four distance matrices. Later, they look for the cells which are neighbors to each other in both 
the directions to define anchors connecting two cells with similar looking expression patterns 
across different single cell datasets. We use the same anchoring algotihm to connect two similar 
cells and define the priors in the bayesian estimation of gene expression.

Anchoring process can sometime generate multi-mapping in both direction, to compensate that we take average 
of all the neighboring cells which has been chosen as anchor to define the prior. In a typical 4000 single 
cell experiment we run alevin and repeat the Seurat3 anchoring approach $30$ times, randomly dividing 
the cells into two equal sets and mapping one set onto the other. We discard the self anchors and use 
anchors with score $>0.5$ as a potential list of neighboring cells to define the priors. This prior 
is then used to optimize the quantification estimates of genes with multi-mapping reads in the alevin 
pipeline.

\section{Results}
\subsection{Comparing quantification methods on simulated data}
\label{subsec:alv2_sims}
Over the years, with the advancement in single cell technologies, various single cell quantification 
pipelines have been proposed. Non availablity of right validation datasets and/or criteria for comparing 
gene-expression estimates generated by the various pipelines has been crucially important. To compare the estimates, we 
had previously proposed an empricial dscRNA-seq data simulation tool, Minnow ~\citep{sarkar2019minnow}. 
Minnow models various features involved in the generation of the dscRNA-seq data like PCR amplification and 
sequencing errors to generate fastq file with the reads and the expeected true cell-v-gene counts. 
We used minnow to simulate a dscRNA-seq experiment with 4340 human PBMC cells with $\sim$ 
20 Million UMI and compare the quantification estimates generated by various methods.

We ran Alevin, Cellranger3 ~\citep{tenx} and bustools pipeline ~\citep{melsted2019modular} on the minnow 
generate reads to compare their quantification estimates with the ground truth. 
In \Cref{tab:matrix_sum,tab:matrix_diff,tab:f1} we show that alevin gives significantly accurate estimates 
compared to other pipelines when we look at the absolute difference of the generated estimates. We observe 
that Cellranger, in expeactation, under expresses a lot of genes resulting in high False Negative (FN) 
while alevin over expressed a lot of genes resulting in higher False Postives (FP). 
The bustools pipeline performed the worse as it reports significantly high FN (almost twice as CR) 
and even the cases where both truth and bustools expresses a gene, their estimates are biased towards 
over expression i.e. low deduplication, provided since they discard gene-multimapping reads their FP 
numbers are the lowest.

\begin{table}[h!]
	\centering
	 \begin{tabular}{|| c | c c c||} 
		 \hline
		 Tools & Matrix Sum & Difference w/ Truth & \% Difference w/ Truth \\ [0.5ex] 
		 \hline\hline
		 Truth & 20,236,199  & 0 & 0 \\ 
		 \hline
		 alevin & 20,619,586 & +383,387 & +1.894\% \\
		 \hline
		 Cellranger & 19,254,114 & -982,085 & -4.853\% \\
		 \hline
		 bustools & 26,499,889 & +6,263,690 & +30.95\% \\ [1ex] 
		 \hline
 	\end{tabular}
	\caption{The comparison of the total number of UMIs as predicted by various tools 
	in comparison to minnow generated true count matrix }
	\label{tab:matrix_sum}
\end{table}

\begin{table}[h!]
	\centering
	 \begin{tabular}{|| c | c||} 
		 \hline
		 Tools & Absolute Difference \\ [0.5ex] 
		 \hline\hline
		 alevin & 869,681 \\
		 \hline
		 Cellranger & 1,656,367 \\
		 \hline
		 bustools & 9,311,578 \\ [1ex] 
		 \hline
 	\end{tabular}
	\caption{The sum of the absolute difference of the tools with the minnow generated 
	true count matrix }
	\label{tab:matrix_diff}
\end{table}

\begin{table}[h!]
	\centering
	 \begin{tabular}{|| c | c | c | | c||} 
		 \hline
		 Metric & alevin & Cellranger & bustools \\ [0.5ex] 
		 \hline\hline
		 False Negative & 44,350 & 271,687 & 400,820 \\
		 \hline
		 False Positive & 260,612 & 54,500 & 6,237 \\ [1ex] 
		 \hline
 	\end{tabular}
	\caption{Comparison of False Discovery metrics where we define False Positive when truth 
	$>$ 0 and method = 0; similarly False Negative when truth = 0 and method $>$ 0}
	\label{tab:f1}
\end{table}

  \begin{figure}[!htb]
      \centering
    \includegraphics[height=3in]{alevin2/fdr.pdf}
    \caption{ Comparison of False Discovery Rate (FDR) i.e. including both FP and FN with the
	the number of mispredicted UMIs. }
    \label{fig:alv2_fdr}
  \end{figure}

  \begin{figure}[!htb]
      \centering
    \includegraphics[height=3in]{alevin2/t3_assign.pdf}
    \caption{ The distribution of the number of genes with the fraction of cells they have tier3
	assignment.}
    \label{fig:alv2_t3}
  \end{figure}

To measure the effect size of false discoveries, we combine the misestimation by all tools
and in ~\Cref{fig:alv2_fdr} we show the relation between the magnitude of misestimation and the fraction 
false discoveries. We include both FP and FN as false discoveries and find that more than 90\% of alevin's 
false predictions are below 1 UMI count. 

\subsection{Variational Bayesian Improves dscRNA-seq quantification}
In ~\Cref{alevin}, we categorize the confidence in the generated gene counts estimates using tiers. 
Even though ~\Cref{subsec:alv2_sims}, we show the alevin generated quants are more accurate
than competing tools, the confidence in tier3 assigned estimates is low and often the source of 
the mis-estimation. We found that in the simulated data, more than 66\% of the false discoveries made 
by alevin are in the tier3 estimates and due to the inherant problem of sparsity in dscRNA-seq data,
we expect the similar pattern of tier3 genes in the real data as well. 

One fundamental property of a dscRNA-seq experiments is that molecules are randomly captured across cells 
and although some cells might randomly sample the read sequence from the ambiguous region 
(or not sample at all) but, in expectation, similar cells in a group can potentially have different 
confidence in their gene estimates. In \Cref{fig:alv2_t3} we show the distribution of the genes with tier3 
prediction in at least one cell, versus the percentage of cells they have tier3 prediction overall.
This motivates our idea that to improve the expression estimates of the tier3 genes, we can utilize the
confidence of neighboring cells and share information across them to improve the gene level estimates 
(details in ~\Cref{sec:alv2_methods}).

\subsubsection{Migration Experiment on Real Data}
We validate that the variational bayesian objective of alevin is more accurate on real data using the
following migration experiment. Alevin pipeline for dscRNA-seq quantification goes through multiple 
phases. After the initial phases of whitelisting and read-mapping, alevin generates an interim 
file format, we call bfh, which contains the transcript equivalence classes along with their Cellular
Barcode and UMI count. A major fraction of tier1 estimates comes from unique transcript equivalence
classes. In our migration experiment, we knocked off (KO) all unique transcript equivalence classes, with
the expectation that the KO bfh will migrate some of the tier1 and tier2 estimates to tier3, as they have
may lose the unique information which made them high confidence tiers (tier1 and tier2) to start with. 

  \begin{figure}[!htb]
      \centering
    \includegraphics[height=3in]{alevin2/migration.png}
    \caption{ Comparison of the cell wise spearman correlation of EM based alevin with VBEM based
	prior enhanced alevin on real data migration experiment}
    \label{fig:alv2_real_val}
  \end{figure}

We took the quantification estimates from pbmc\_4k experiment with 4340 cells and randomly divide it into
two parts of 2170 cells each, we call 2ka and 2kb experiment. We perform the migration experiment on
2ka cells and generate alevin EM estimates using KO bfh. Later, we anchors KO 2ka estimates 
with full 2kb estimates and learn informative priors for KO 2ka estimates to requantify using alevin VBEM
approach. In ~\Cref{fig:alv2_real_val}, we show the cell wise correlaion of KO 2ka experiment (as em) and prior
enhanced version of alevin (as vbem). Since the anchoring scheme can't always find high confidence anchors for 
all cells, we see a bimodal distribution with vbem approach which signifies the first mode (overlapping 
with em) are cells with no informative prior while the second mode (with imporved correlation)
signifies cells with informed prior.

\subsubsection{VBEM Validation on Simulated Data}
We validate our prior enhanved vbem approach on simulated data using the minnow generated data (details in
~\Cref{subsec:alv2_sims}). We first use em based alevin on the simulated data to generate quantification
estimates on 4340 cells. We use Seurat's anchoring scheme to self anchor the estimates and generate the
prior. Using the learnt prior we requantify the simulated dataset with our vbem based approach to generate 
the new estimates. In ~\Cref{fig:alv2_sim_val}, we compare the cell wise correlation of both em and vbem
based approaches to the minnow generated ground truth and observe higher correlation with vbem generated
gene expression estimates.

  \begin{figure}[!htb]
      \centering
    \includegraphics[height=3in]{alevin2/sim_corr.pdf}
    \caption{ Comparison of the cell wise spearman correlation of em based alevin with vbem based
	prior enhanced alevin on simulated experiment}
    \label{fig:alv2_sim_val}
  \end{figure}

\section{Conclusion}
We proposed a bayesian framework which extends previous generation alevin's maximum likelihood based 
quantification procedure. We explore different techniques of generating priors and show that our 
information sharing framework consistently improves the tier3 dscRNA-seq quantification estimates and 
is especially useful for highly ambiguous estimates where there is no intra cellular activity to 
effiiently quantify. We plan to extend the framework towards incorporating priors from new upcoming 
technologies, for example spatial data can be substantially useful for setting the prior in the proposed 
alevin framework.  We think this framework has potential to open a new direction of multi-modal 
quantification of the data and the community will adopt and help improve it further.
