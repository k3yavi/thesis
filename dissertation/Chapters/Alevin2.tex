% Chapter 1

\chapter{Alevin 2} % Main chapter title

\label{alevin2} % For referencing the chapter elsewhere, use \ref{Chapter1} 

%----------------------------------------------------------------------------------------

\section{Background}

RNA-seq quantification has been shown as an important tool to explore the genome-wide
quantification of the gene expressions for both bulk and single cell RNA-seq experiments.
Specially dscRNA-seq, recently with the advancement in droplet-based sequencing technologies
~\citet{dropseq, indrop, tenx} is generating data with high quantitative accuracy,
sensitivity, and throughput. In ~\Cref{alevin}, we proposed a principled framework for 
generating accurate gene-expression estimate. We showed how discarding gene-multimapping 
reads by all the other dscRNA-seq quantification pipelines leads to biased expression estimates. 

In previous generation of alevin, after the UMI resolution phase, the ambiguous reads with more 
than 1 gene label are assigned to genes based on an expectation-maximization method. The 
framework works well in most of the cases, however, in situations where there is no unique 
read evidence to confidently disambiguate the read label among the genes, we called them 
tier-3 estimates, the likelihood estimator uniformly divides the reads across the genes in the 
label. In this study we propose the idea of information sharing across closely related cells 
using bayesian prior for specifically improving the tier-3 estimates. Intuitively, since we expect 
relatively similar gene expression among the cells with similar types, we propose that sharing 
information (eg. which genes are expressed) across neighboring cells improves the quantification 
estimates of an individual cells. We show that the novel idea of information sharing using bayesian 
prior to improve the quantification can be successfully applied to a dscRNA-seq data for improving the 
quantification estimates.

\section{Material and Methods}
\label{sec:alv2_methods}

\section{Results}
\subsection{Comparing quantification methods on simulated data}
\label{subsec:alv2_sims}
Over the years, with the advancement in single cell technologies, various single cell quantification 
pipelines have been proposed. Non availablity of right validation datasets and/or criteria for comparing 
gene-expression estimates generated by the pipelines have been crucial. To compare the estimates, we 
had previously proposed an empricial dscRNA-seq data simulation tool, Minnow ~\citep{sarkar2019minnow}. 
Minnow models various features involved in the generation of the dscRNA-seq data like PCR amplification, 
sequencing errors to generate fastq file with the reads and the expeected true cell-v-gene counts post 
quantification. We used minnow to simulate a dscRNA-seq experiment with 4000 human PBMC cells with $\sim$ 
20 Million UMI and compare the quantification estimates generated by various methods.

We ran Alevin, Cellranger3 ~\citep{tenx} and bustools pipeline ~\citep{melsted2019modular} on the minnow 
generate reads to compare their quantification estimates with the ground truth. 
In \Cref{tab:matrix_sum,tab:matrix_diff,tab:f1} we show that alevin gives significantly accurate estimates 
compared to other pipelines when we look at the absolute difference of the generated estimates. We observe 
that Cellranger, in expeactation, under expresses a lot of genes resulting in high False Negative (FN) 
while alevin over expressed a lot of genes resulting in higher False Postives (FP). 
The bustools pipeline performed the worse as it reports significantly high FN (almost twice as CR) 
and even the cases where both truth and bustools expresses a gene, their estimates are biased towards 
over expression i.e. low deduplication, provided their FP numbers are the lowest.

\begin{table}[h!]
	\centering
	 \begin{tabular}{|| c | c c c||} 
		 \hline
		 Tools & Matrix Sum & Difference w/ Truth & \% Difference w/ Truth \\ [0.5ex] 
		 \hline\hline
		 Truth & 20,236,199  & 0 & 0 \\ 
		 \hline
		 alevin & 20,619,586 & +383,387 & +1.894\% \\
		 \hline
		 Cellranger & 19,254,114 & -982,085 & -4.853\% \\
		 \hline
		 bustools & 26,499,889 & +6,263,690 & +30.95\% \\ [1ex] 
		 \hline
 	\end{tabular}
	\caption{The comparison of the total number of UMIs as predicted by various tools 
	in comparison to minnow generated true count matrix }
	\label{tab:matrix_sum}
\end{table}

\begin{table}[h!]
	\centering
	 \begin{tabular}{|| c | c||} 
		 \hline
		 Tools & Absolute Difference \\ [0.5ex] 
		 \hline\hline
		 alevin & 869,681 \\
		 \hline
		 Cellranger & 1,656,367 \\
		 \hline
		 bustools & 9,311,578 \\ [1ex] 
		 \hline
 	\end{tabular}
	\caption{The sum of the absolute difference of the tools with the minnow generated 
	true count matrix }
	\label{tab:matrix_diff}
\end{table}

\begin{table}[h!]
	\centering
	 \begin{tabular}{|| c | c | c | | c||} 
		 \hline
		 Metric & alevin & Cellranger & bustools \\ [0.5ex] 
		 \hline\hline
		 False Negative & 44,350 & 271,687 & 400,820 \\
		 \hline
		 False Positive & 260,612 & 54,500 & 6,237 \\ [1ex] 
		 \hline
 	\end{tabular}
	\caption{Comparison of False Discovery metrics where we define False Positive when truth 
	$>$ 0 and method = 0; similarly False Negative when truth = 0 and method $>$ 0}
	\label{tab:f1}
\end{table}

\subsection{Simulations and Real data shares similar patterns of tier3 genes}
In our previously proposed framework we categorize the confidence in the generated gene counts 
estimates using tiers. Even though in previous section we show the alevin generated quants are 
more accurate, the confidence in tier3 assigned estimates is low and often the cause of the 
accuracy difference with the truth. We expect the similar pattern of tier3 genes in the real 
data as well. 
%\avi{add a figure comparing tiers of real data w/ simulated}

\subsection{Cross Sample Cellular Barcode Anchoring}
Tier 3 estimates are specifically the estimates which have reads assigned to gene ambiguous 
labels and they don't have any uniquely assignable read evidence, even in their equivalence 
class network. To disambiguate the labels for the assignable gene we propose to use the 
information either from the cells with similar type within the sample or across different 
type of data itself such as ATAC-seq data from similar tissue. To find similar cells for sharing 
the confidence in their gene expression estimates we use Seurat ~\citep{seurat3} based anchoring 
scheme. In Seurat3 stuart et. al. proposed an anchoring scheme in which they defined a framework 
to connect two experiments based on the similarity in various cell's gene expression pattern. 
They first find nearest neighbor using l2 distance of both inter and intra datatsets to generate 
four distance matrices. Later, they look for the cells which are neighbors to each other in both 
the directions to define anchors connecting two cells with similar looking expression patterns 
across different single cell datasets. We use the same anchoring algotihm to connect two similar 
cells and define the priors in their quantification estimation.

\subsection{Variational Bayesian Improves dscRNA-seq quantification}
The previously proposed framework of alevin esimates the maximum likelihood estimates (MLE) to 
disambiguate the gene ambiguous read assignment, however, it lacks the ability to utilize the 
confidence from neighboring cells. Salmon ~\citep{salmon} a bulk RNA-seq quantification tools uses 
bayesian priors to improve the quantification estimates by its online learning phase. We use the 
relaxed version of the algorithm (See \Cref{sec:alv2_methods}) to improve the estimates of alevin. 
We use the anchoring scheme to connect cells in two datasets and use the estimates from one as a 
prior to second for providing the confidence in a gene's expression within a cell. Anchoring process 
can sometime generate multi-mapping in both direction, to compensate that we take average of all 
the neighboring cells which has been chosen as anchor to define the prior. In a typical 4000 single 
cell experiment we run alevin and repeat the Seurat3 anchoring approach $30$ times, randomly dividing 
the cells into two equal sets and mapping one set onto the other. We discard the self anchors and use 
anchors with score $>0.5$ as a potential list of neighboring cells to define the priors. This prior 
is then used to optimize the quantification estimates of genes with multi-mapping reads in the alevin 
pipeline.

\section{Conclusion}
We proposed a bayesian framework which extends previous generation alevin's maximum likelihood based 
quantification procedure. We explore different techniques of generating priors and show that our 
information sharing framework consistently improves the tier3 dscRNA-seq quantification estimates and 
is especially useful for highly ambiguous estimates where there is no intra cellular activity to 
effiiently quantify. We plan to extend the framework towards incorporating priors from new upcoming 
technologies, for example spatial data can be substantially useful for setting the prior in the proposed 
alevin framework.  We think this framework has potential to open a new direction of multi-modal 
quantification of the data and the community will adopt and help improve it further.