% Chapter 1

\chapter{Discussion and Conclusion} % Main chapter title
\label{conclusion} % For referencing the chapter elsewhere, use \ref{Chapter1} 

%----------------------------------------------------------------------------------------

% Define some commands to keep the formatting separated from the content 
% \newcommand{\keyword}[1]{\textbf{#1}}
% \newcommand{\tabhead}[1]{\textbf{#1}}
% \newcommand{\code}[1]{\texttt{#1}}
% \newcommand{\file}[1]{\texttt{\bfseries#1}}
% \newcommand{\option}[1]{\texttt{\itshape#1}}

In the following sections, we give a brief overview of the efficient computational 
methods this study presents which improve the quantification of both bulk and dscRNA-seq 
data along with some of the supplementary work and the future directions.

\section{Bulk RNA-seq}

In ~\Cref{rapmap} we discuss the usefulness of our novel approach, \qm, 
for mapping RNA-seq reads. We show our efficient implementation of \qm, in the 
form of \rapmap is fast, accurate and addresses an important problem of read-mapping.
In ~\Cref{selective_alignment} we discuss the extension of our \qm approach into
\hsa, we discuss the pitfalls of using only the simulated dataset for validating 
computational methods and how including genomic alignment as decoys for the
transcriptomic alignment improves the performance of transcript abundance estimation.
In addition to that, we are reaching out to find other biological applications where 
\hsa can be useful. In the following section, we discuss some of the other applications, 
which we have worked on. 

\subsection{Supplementary Study}
\subsubsection{RapClust~\citep{srivastava2016accurate}}

De novo transcriptome assembly of non-model organisms is the first major step for many 
RNA-seq analysis tasks. Current methods for de novo assembly often report a large number 
of contiguous sequences (contigs), which may be fractured and incomplete sequences 
instead of full-length transcripts. Dealing with a large number of such contigs can 
slow and complicate downstream analysis. In this extension study, we present a method 
for clustering contigs from de novo transcriptome assemblies based upon the relationships 
exposed by multi-mapping sequencing fragments. Specifically, we cast the problem of 
clustering contigs as one of clustering a sparse graph that is induced by equivalence 
classes of fragments that map to subsets of the transcriptome. Leveraging recent developments 
in efficient read mapping and transcript quantification, we have developed RapClust, a tool 
implementing this approach that is capable of accurately clustering most large de novo 
transcriptomes in a matter of minutes, while simultaneously providing accurate estimates of 
expression for the resulting clusters. We compare RapClust against several tools commonly 
used for de novo transcriptome clustering. Using de novo assemblies of organisms for which 
reference genomes are available, we assess the accuracy of these different methods in terms of 
the quality of the resulting clusterings, and the concordance of differential expression tests 
with those based on ground truth clusters. We find that RapClust produces clusters of comparable 
or better quality than existing state-of-the-art approaches, and does so substantially faster. 
RapClust also confers a large benefit in terms of space usage, as it produces only succinct 
intermediate files - usually on the order of a few megabytes - even when processing hundreds 
of millions of reads.

\section{Single cell RNA-seq}
In ~\cref{alevin}, we present a new end-to-end pipeline for performing gene-level 
quantification from \dscrnaseq. We show that, compared to other methods, \alevin achieves
 higher accuracy, in part because of considering a substantially larger number
of reads. Further, \alevin is considerably faster and uses less memory than
these other approaches. The optimizations used in \alevin makes it possible to 
efficiently process \dscrnaseq datasets in a reduced computational burden for 
processing and re-processing of \dscrnaseq data. 
In ~\cref{alevin2}, we present the extension of \alevin into using a bayesian
approach for solving the expectation-maximization objective. We show the new
approach is specifically useful for tier3 (high ambiguous) estimates and 
significantly improves the cell-wise gene expression estimates by sharing 
the information across cells. In addition to that, we are reaching out to 
find other \singlecell platforms where \alevin can be useful. 
In the following section, we discuss some of the applications, which we have worked on, 
or are currently working.

\subsection{Supplementary Study}
\subsubsection{Swish ~\citep{zhu2019nonparametric}}
A primary challenge in the analysis of RNA-seq data is to identify differentially 
expressed genes or transcripts while controlling for technical biases present in 
the observations. Ideally, a statistical testing procedure should incorporate 
information about the inherent uncertainty of the abundance estimates, whether at 
the gene or transcript level, that arise from quantification of abundance. Most 
popular methods for RNA-seq differential expression analysis fit a parametric model 
to the counts or scaled counts for each gene or transcript, and a subset of methods 
can incorporate information about the uncertainty of the counts. Previous work has 
shown that nonparametric models for RNA-seq differential expression may in some cases 
have better control of the false discovery rate, and adapt well to new data types 
without requiring reformulation of a parametric model. Existing nonparametric models 
do not take into account the inferential uncertainty of the observations, leading to 
an inflated false discovery rate, in particular at the transcript level. Here we 
propose a nonparametric model for differential expression analysis using inferential 
replicate counts, extending the existing SAMseq method to account for inferential 
uncertainty, batch effects, and sample pairing. We compare our method, 
"SAMseq With Inferential Samples Helps", or Swish, with popular differential 
expression analysis methods. Swish has improved control of the false discovery rate, 
in particular for transcripts with high inferential uncertainty. We apply Swish to 
a single-cell RNA-seq dataset, assessing sensitivity to recover DE genes between 
sub-populations of cells, and compare its performance to the Wilcoxon rank sum test.

\subsubsection{Minnow ~\citep{sarkar2019minnow}}
With the advancements of high-throughput single-cell RNA-sequencing protocols, there 
has been a rapid increase in the tools available to perform an array of analyses on the 
gene expression data that results from such studies.  For example, there exist methods 
for pseudo-time series analysis, differential cell usage ,cell-type detection RNA-velocity 
in single cells etc. Most analysis pipelines validate their results using known marker 
genes (which are not widely available for all types of analysis) and by using simulated 
data from gene-count-level simulators. Typically, the impact of using different 
read-alignment or UMI deduplication methods has not been widely explored. Assessments 
based on simulation tend to start at the level of assuming a simulated count matrix, 
ignoring the effect that different approaches for resolving UMI counts from the raw read 
data may produce. Here, we present minnow, a comprehensive sequence-level droplet-based 
single-cell RNA-seq (dscRNA-seq) experiment simulation framework.  Minnow accounts for 
important sequence-level characteristics of experimental scRNA-seq datasets and models 
effects such as PCR amplification,  CB (cellular barcodes) and UMI (Unique Molecule 
Identifiers) selection, and sequence fragmentation and sequencing. It also closely matches 
the gene-level ambiguity characteristics that are observed in real scRNA-seq experiments.  
Using minnow, we explore the performance of some common processing pipelines to produce 
gene-by-cell count matrices from droplet-bases scRNA-seq data, demonstrate the effect that 
realistic levels of gene-level sequence ambiguity can have on accurate quantification, 
and show a typical use-case of minnow in assessing the output generated by different 
quantification pipelines on the simulated experiment.
